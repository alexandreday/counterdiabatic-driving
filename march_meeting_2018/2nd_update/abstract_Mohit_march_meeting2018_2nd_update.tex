\documentclass[11pt,a4paper]{article}
\usepackage[utf8]{inputenc}
\usepackage[hmargin=2.0cm,vmargin=2.5cm,bindingoffset=0.5cm]{geometry}
\usepackage{amsfonts}
\usepackage{amsmath,amsthm,amssymb}
\usepackage[english]{babel}
\author{Mohit Pandey, Dries Sels and David. K. Campbell}
\title{Approximate counter-diabatic driving protocols for non-integrable quantum systems  }
\begin{document}
\maketitle
\textbf{11.0 STRONGLY CORRELATED SYSTEMS, INCLUDING QUANTUM FLUIDS AND SOLIDS (DCMP): 11.1.6 Non-Equilibrium Physics with Cold Atoms and Molecules, Rydberg Gases, and Trapped Ions (DAMOP, DCMP)}

Noise and decoherence caused by the environment are two major challenges in applying adiabatic protocols to quantum technologies. Counter-diabatic (CD) driving protocols, which are also known as "shortcuts-to-adiabaticity," provide powerful alternatives for controlling a quantum system. These protocols allow one to change Hamiltonian parameters rapidly while still mimicking adiabatic dynamics. They have been shown to work well for a wide variety of systems, but it is exponentially hard to find exact CD protocols for non-integrable quantum many-body systems. We study a method to develop approximate CD protocols which avoids exponential sensitivity to perturbations of the Hamiltonian. Our finite-size scaling of CD Hamiltonians reveals remarkable differences between integrable and non-integrable quantum systems. We identify  numerically different scaling regimes and show how they arise from the eigenstate thermalization hypothesis.

\end{document}
